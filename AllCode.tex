\documentclass[12 pt]{article}
\usepackage{fullpage}
\usepackage{amsthm}
\usepackage{amsmath}
\usepackage{xcolor}
\usepackage{amssymb}

\newcommand{\new}[1]{\bigskip \noindent \color{blue} \textbf{#1}\medskip}
\newcommand{\q}[1]{\response{\emph{#1}}}
\newcommand{\prompt}[1]{\noindent \textbf{#1}}
\newcommand{\response}[1]{\color{black} #1 \medskip}
\newcommand{\bb}[1]{\mathbb{#1}}
\newcommand{\pset}[1]{\mathcal{P}({#1})}

\newcommand{\Q}[2]{\new{Question {#1}:} \q{#2}}
\newcommand{\proposition}[1]{\prompt{Proposition:} \response{#1}}
\newcommand{\discussion}[2]{\\ \prompt{Discussion:} \begin{itemize}
    \item \prompt{What we know:} {#1}
    \item \prompt{What we'll do:} {#2}
\end{itemize}}
\newcommand{\p}[1]{\begin{proof} \response{{#1}} \end{proof}}
\newcommand{\con}[1]{\\ \prompt{Conjecture:}\response{#1}}

\begin{document}

Final Code submission, Edward Speer, EE/CS10b Spring '23
\tiny
\begin{verbatim}

;;;;;;;;;;;;;;;;;;;;;;;;;;;;;;;;;;;;;;;;;;;;;;;;;;;;;;;;;;;;;;;;;;;;;;;;;;;;;;;;;;;;;;;;;;;;;;;;;;;;;;;;;;;;;;;;;;;;;;;;;;;
;                                                                                                                         ;
;                          				                   HEXER GAME                                                                  ;
;                          			                 Functional Specification                                                       ;
;                                                           EE/CS 10b                                                   ;
;                                                                                                                         ;
;;;;;;;;;;;;;;;;;;;;;;;;;;;;;;;;;;;;;;;;;;;;;;;;;;;;;;;;;;;;;;;;;;;;;;;;;;;;;;;;;;;;;;;;;;;;;;;;;;;;;;;;;;;;;;;;;;;;;;;;;;;


  Description:  	The system is a Hexer game in which the user uses 5 switches to attempt to manipulate  
			the game configuration displayed on LEDs into the winning configuration via 5 legal moves.
			There is 7 segment display for indicating the current move number, and reset switches which
			allow both manual reset to a given configuration or game reset to a random configuration. 
			On game start there is a speaker which begins playing game music, and upon winning the game, 
			the speaker switches to a different song and display blinks. In order to win the game, the user 
			manipulates the game board via the 5 legal moves until all of the LEDs in the 15 LED display
			are turned ON.

  Global Variables:	GameBoard - The configuration of the game board at any time throughout the game play.


  Inputs:                All user input occurs through 6 switches:
			 _________________________________________________________________________________________
                        |  Switch Name  |  Switch Type  |                        Description                      |
			|---------------|---------------|---------------------------------------------------------|
			|     START     |  Pushbutton   | Used to start a new game. Must be held down while one of|
			|		|		| either "Random Reset", "Manual Reset 1", or "Manual     |
			|		|		| Reset 0" is selected to indicate whether a random or    |
			|		|		| custom configuration is desired.			  |
			|---------------|---------------|---------------------------------------------------------|
			|   Move 1 /    |  Pushbutton   | When pressed during game play, this switch executes an  |
			|    (White)	|        	| inversion of the inner hexagon on the game board, i.e   |
			|               |		| inversion of the states of LEDs 3, 4, 7, 9, 12, and 13. |
			|		|		|							  |
			|---------------|---------------|---------------------------------------------------------|
			|   Move 2 /    |  Pushbutton   | When pressed during game play, this switch executes a   |
			|  Random Reset |		| clockwise rotation of the right hexagon on the game     |
			|    (Black)    |		| board consisting of LEDs 4, 5, 10, 14, 13, and 8        |
			|		|		|							  |
			|		|		| When pressed while the START switch is held, this switch|
			|		|		| causes the move number to reset and the LEDs to take on |
			|		|		| a random configuration.				  |
			|---------------|---------------|---------------------------------------------------------|
			|   Move 3 /    |  Pushbutton   | When pressed during game play, this switch executes a   |  
			| Manual Reset 1|		| counterclockwise rotation of the states of the left     |
			|    (Blue)     |		| hexagon in the game board, consisting of LEDS 2, 3, 8,  |
			|		|		| 12, 11, and 6.					  |
			|		|		|							  |
			|		|		| When pressed while the START button is held, this switch|
			|		|		| shifts an ON LED into the game configuration, which acts|
			|		|		| as a shift register in manual reset mode.		  |
			|---------------|---------------|---------------------------------------------------------|
			|    Move 4     |  Pushbutton   | When pressed, this switch executes a clockwise rotation |
			|     (Red)     |		| of the upper triangle in the game board, consisting of  |
			|		|		| LEDS 1, 4, 9, 14, 13, 12, 11, 7, and 3.                 |
			|---------------|---------------|---------------------------------------------------------|
			|   Move 5/     |  Pushbutton   | When pressed this switch executes a counterclockwise    |
			| Manual Reset 0|		| rotation of the lower triangle in the game board,       |
			|    (Green)	|		| consisting of LEDs 2, 3, 4, 5, 9, 13, 15, 12, and 7.    |
			|		|		|							  |
			|		|		| When pressed while the START switch is held, this switch|
			|		|		| resets the move number, and shifts a 0 into the game LED|
			|		|		| board, with all LEDs treated as a shift register.       |
			|_______________|_______________|_________________________________________________________|


  Outputs:		4-digit 7-segment display:  Displays the current number of moves used by the player in
						    attempting to solve the puzzle since last resetting using the 
						    START button.

			15-LED display:             Displays the state of the current game board. Flashes when the 
						    user has solved the puzzle and won the game.

			Speaker:                    Plays game music throughout the game place (Star Trek theme 
                                                    music), then upon winning, plays the Mario Stage Win music 
						    until the user resets the game, then returns to the game play
						    music until the user wins again. 
			

  User Interface: 	The current game state is displayed on the 15 LEDs. The current number of moves used by 
			the player is displayed on the 4-digit 7 segment LED display. The user can use any of the 5
			switches corresponding to moves to manipulate the LED's in the game state as described above
			in "inputs". At any time, the user may reset the game state and the moves counter by holding  
			the START button, and pressing one of the random or manual reset switches. When these are 
                        pressed at the same time as the start switch they will reset the game as described in the 
                        "inputs" section above. Any time a user wins the game, they are informed of their victory 
			by the blinking of the game board LEDs and the playing of stage win music.

  Error Handling:	If there is a power failure, the system will reset to a psuedo-random configuration of the
			game board, with the 4-digit display showing 0 moves.

			If a player presses the START button but does not select one of either the "Random Reset"
			or the "Manual Reset _" switches, then the button will have no effect. The 4 digit display
			and 15 LED displays will continue to display the data of the current game session.

			If the move number used by the player overflows the 4 digit display, the system automatically 
			resets the move counter back to 0.

			The system's switches must be appropriately debounced so that inconsistencies in the 
			hardware do not cause any disruption or change in game control that users do not intend.


  Limitations:		If a player uses more moves than are able to fit in the 4 digit display (9999), the game must 
			reset, so that there is a limit to the number of moves that a game session can be played 
			for.

\end{verbatim}
\end{document}